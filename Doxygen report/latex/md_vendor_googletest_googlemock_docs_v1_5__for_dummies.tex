({\bfseries{Note\+:}} If you get compiler errors that you don\textquotesingle{}t understand, be sure to consult Google Mock Doctor.)

\section*{What Is Google C++ Mocking Framework?}

When you write a prototype or test, often it\textquotesingle{}s not feasible or wise to rely on real objects entirely. A {\bfseries{mock object}} implements the same interface as a real object (so it can be used as one), but lets you specify at run time how it will be used and what it should do (which methods will be called? in which order? how many times? with what arguments? what will they return? etc).

{\bfseries{Note\+:}} It is easy to confuse the term {\itshape fake objects} with mock objects. Fakes and mocks actually mean very different things in the Test-\/\+Driven Development (T\+DD) community\+:


\begin{DoxyItemize}
\item {\bfseries{Fake}} objects have working implementations, but usually take some shortcut (perhaps to make the operations less expensive), which makes them not suitable for production. An in-\/memory file system would be an example of a fake.
\item {\bfseries{Mocks}} are objects pre-\/programmed with {\itshape expectations}, which form a specification of the calls they are expected to receive.
\end{DoxyItemize}

If all this seems too abstract for you, don\textquotesingle{}t worry -\/ the most important thing to remember is that a mock allows you to check the {\itshape interaction} between itself and code that uses it. The difference between fakes and mocks will become much clearer once you start to use mocks.

{\bfseries{Google C++ Mocking Framework}} (or {\bfseries{Google \mbox{\hyperlink{class_mock}{Mock}}}} for short) is a library (sometimes we also call it a \char`\"{}framework\char`\"{} to make it sound cool) for creating mock classes and using them. It does to C++ what \href{http://www.jmock.org/}\texttt{ j\+Mock} and \href{http://www.easymock.org/}\texttt{ Easy\+Mock} do to Java.

Using Google \mbox{\hyperlink{class_mock}{Mock}} involves three basic steps\+:


\begin{DoxyEnumerate}
\item Use some simple macros to describe the interface you want to mock, and they will expand to the implementation of your mock class;
\end{DoxyEnumerate}
\begin{DoxyEnumerate}
\item Create some mock objects and specify its expectations and behavior using an intuitive syntax;
\end{DoxyEnumerate}
\begin{DoxyEnumerate}
\item Exercise code that uses the mock objects. Google \mbox{\hyperlink{class_mock}{Mock}} will catch any violation of the expectations as soon as it arises.
\end{DoxyEnumerate}

\section*{Why Google \mbox{\hyperlink{class_mock}{Mock}}?}

While mock objects help you remove unnecessary dependencies in tests and make them fast and reliable, using mocks manually in C++ is {\itshape hard}\+:


\begin{DoxyItemize}
\item Someone has to implement the mocks. The job is usually tedious and error-\/prone. No wonder people go great distance to avoid it.
\item The quality of those manually written mocks is a bit, uh, unpredictable. You may see some really polished ones, but you may also see some that were hacked up in a hurry and have all sorts of ad hoc restrictions.
\item The knowledge you gained from using one mock doesn\textquotesingle{}t transfer to the next.
\end{DoxyItemize}

In contrast, Java and Python programmers have some fine mock frameworks, which automate the creation of mocks. As a result, mocking is a proven effective technique and widely adopted practice in those communities. Having the right tool absolutely makes the difference.

Google \mbox{\hyperlink{class_mock}{Mock}} was built to help C++ programmers. It was inspired by \href{http://www.jmock.org/}\texttt{ j\+Mock} and \href{http://www.easymock.org/}\texttt{ Easy\+Mock}, but designed with C++\textquotesingle{}s specifics in mind. It is your friend if any of the following problems is bothering you\+:


\begin{DoxyItemize}
\item You are stuck with a sub-\/optimal design and wish you had done more prototyping before it was too late, but prototyping in C++ is by no means \char`\"{}rapid\char`\"{}.
\item Your tests are slow as they depend on too many libraries or use expensive resources (e.\+g. a database).
\item Your tests are brittle as some resources they use are unreliable (e.\+g. the network).
\item You want to test how your code handles a failure (e.\+g. a file checksum error), but it\textquotesingle{}s not easy to cause one.
\item You need to make sure that your module interacts with other modules in the right way, but it\textquotesingle{}s hard to observe the interaction; therefore you resort to observing the side effects at the end of the action, which is awkward at best.
\item You want to \char`\"{}mock out\char`\"{} your dependencies, except that they don\textquotesingle{}t have mock implementations yet; and, frankly, you aren\textquotesingle{}t thrilled by some of those hand-\/written mocks.
\end{DoxyItemize}

We encourage you to use Google \mbox{\hyperlink{class_mock}{Mock}} as\+:


\begin{DoxyItemize}
\item a {\itshape design} tool, for it lets you experiment with your interface design early and often. More iterations lead to better designs!
\item a {\itshape testing} tool to cut your tests\textquotesingle{} outbound dependencies and probe the interaction between your module and its collaborators.
\end{DoxyItemize}

\section*{Getting Started}

Using Google \mbox{\hyperlink{class_mock}{Mock}} is easy! Inside your C++ source file, just {\ttfamily \#include} {\ttfamily $<$\mbox{\hyperlink{gtest_8h_source}{gtest/gtest.\+h}}$>$} and {\ttfamily $<$\mbox{\hyperlink{gmock_8h_source}{gmock/gmock.\+h}}$>$}, and you are ready to go.

\section*{A Case for \mbox{\hyperlink{class_mock}{Mock}} Turtles}

Let\textquotesingle{}s look at an example. Suppose you are developing a graphics program that relies on a L\+O\+G\+O-\/like A\+PI for drawing. How would you test that it does the right thing? Well, you can run it and compare the screen with a golden screen snapshot, but let\textquotesingle{}s admit it\+: tests like this are expensive to run and fragile (What if you just upgraded to a shiny new graphics card that has better anti-\/aliasing? Suddenly you have to update all your golden images.). It would be too painful if all your tests are like this. Fortunately, you learned about Dependency Injection and know the right thing to do\+: instead of having your application talk to the drawing A\+PI directly, wrap the A\+PI in an interface (say, {\ttfamily Turtle}) and code to that interface\+:


\begin{DoxyCode}{0}
\DoxyCodeLine{class Turtle \{}
\DoxyCodeLine{  ...}
\DoxyCodeLine{  virtual ~Turtle() \{\}}
\DoxyCodeLine{  virtual void PenUp() = 0;}
\DoxyCodeLine{  virtual void PenDown() = 0;}
\DoxyCodeLine{  virtual void Forward(int distance) = 0;}
\DoxyCodeLine{  virtual void Turn(int degrees) = 0;}
\DoxyCodeLine{  virtual void GoTo(int x, int y) = 0;}
\DoxyCodeLine{  virtual int GetX() const = 0;}
\DoxyCodeLine{  virtual int GetY() const = 0;}
\DoxyCodeLine{\};}
\end{DoxyCode}


(Note that the destructor of {\ttfamily Turtle} {\bfseries{must}} be virtual, as is the case for {\bfseries{all}} classes you intend to inherit from -\/ otherwise the destructor of the derived class will not be called when you delete an object through a base pointer, and you\textquotesingle{}ll get corrupted program states like memory leaks.)

You can control whether the turtle\textquotesingle{}s movement will leave a trace using {\ttfamily Pen\+Up()} and {\ttfamily Pen\+Down()}, and control its movement using {\ttfamily Forward()}, {\ttfamily Turn()}, and {\ttfamily Go\+To()}. Finally, {\ttfamily Get\+X()} and {\ttfamily Get\+Y()} tell you the current position of the turtle.

Your program will normally use a real implementation of this interface. In tests, you can use a mock implementation instead. This allows you to easily check what drawing primitives your program is calling, with what arguments, and in which order. Tests written this way are much more robust (they won\textquotesingle{}t break because your new machine does anti-\/aliasing differently), easier to read and maintain (the intent of a test is expressed in the code, not in some binary images), and run {\itshape much, much faster}.

\section*{Writing the \mbox{\hyperlink{class_mock}{Mock}} Class}

If you are lucky, the mocks you need to use have already been implemented by some nice people. If, however, you find yourself in the position to write a mock class, relax -\/ Google \mbox{\hyperlink{class_mock}{Mock}} turns this task into a fun game! (Well, almost.)

\subsection*{How to Define It}

Using the {\ttfamily Turtle} interface as example, here are the simple steps you need to follow\+:


\begin{DoxyEnumerate}
\item Derive a class {\ttfamily Mock\+Turtle} from {\ttfamily Turtle}.
\end{DoxyEnumerate}
\begin{DoxyEnumerate}
\item Take a virtual function of {\ttfamily Turtle}. Count how many arguments it has.
\end{DoxyEnumerate}
\begin{DoxyEnumerate}
\item In the {\ttfamily public\+:} section of the child class, write {\ttfamily M\+O\+C\+K\+\_\+\+M\+E\+T\+H\+O\+Dn();} (or {\ttfamily M\+O\+C\+K\+\_\+\+C\+O\+N\+S\+T\+\_\+\+M\+E\+T\+H\+O\+Dn();} if you are mocking a {\ttfamily const} method), where {\ttfamily n} is the number of the arguments; if you counted wrong, shame on you, and a compiler error will tell you so.
\end{DoxyEnumerate}
\begin{DoxyEnumerate}
\item Now comes the fun part\+: you take the function signature, cut-\/and-\/paste the {\itshape function name} as the {\itshape first} argument to the macro, and leave what\textquotesingle{}s left as the {\itshape second} argument (in case you\textquotesingle{}re curious, this is the {\itshape type of the function}).
\end{DoxyEnumerate}
\begin{DoxyEnumerate}
\item Repeat until all virtual functions you want to mock are done.
\end{DoxyEnumerate}

After the process, you should have something like\+:


\begin{DoxyCode}{0}
\DoxyCodeLine{\#include <gmock/gmock.h>  // Brings in Google Mock.}
\DoxyCodeLine{class MockTurtle : public Turtle \{}
\DoxyCodeLine{ public:}
\DoxyCodeLine{  ...}
\DoxyCodeLine{  MOCK\_METHOD0(PenUp, void());}
\DoxyCodeLine{  MOCK\_METHOD0(PenDown, void());}
\DoxyCodeLine{  MOCK\_METHOD1(Forward, void(int distance));}
\DoxyCodeLine{  MOCK\_METHOD1(Turn, void(int degrees));}
\DoxyCodeLine{  MOCK\_METHOD2(GoTo, void(int x, int y));}
\DoxyCodeLine{  MOCK\_CONST\_METHOD0(GetX, int());}
\DoxyCodeLine{  MOCK\_CONST\_METHOD0(GetY, int());}
\DoxyCodeLine{\};}
\end{DoxyCode}


You don\textquotesingle{}t need to define these mock methods somewhere else -\/ the {\ttfamily M\+O\+C\+K\+\_\+\+M\+E\+T\+H\+O\+D$\ast$} macros will generate the definitions for you. It\textquotesingle{}s that simple! Once you get the hang of it, you can pump out mock classes faster than your source-\/control system can handle your check-\/ins.

{\bfseries{Tip\+:}} If even this is too much work for you, you\textquotesingle{}ll find the {\ttfamily gmock\+\_\+gen.\+py} tool in Google \mbox{\hyperlink{class_mock}{Mock}}\textquotesingle{}s {\ttfamily scripts/generator/} directory (courtesy of the \href{http://code.google.com/p/cppclean/}\texttt{ cppclean} project) useful. This command-\/line tool requires that you have Python 2.\+4 installed. You give it a C++ file and the name of an abstract class defined in it, and it will print the definition of the mock class for you. Due to the complexity of the C++ language, this script may not always work, but it can be quite handy when it does. For more details, read the \href{http://code.google.com/p/googlemock/source/browse/trunk/scripts/generator/README}\texttt{ user documentation}.

\subsection*{Where to Put It}

When you define a mock class, you need to decide where to put its definition. Some people put it in a {\ttfamily $\ast$\+\_\+test.cc}. This is fine when the interface being mocked (say, {\ttfamily Foo}) is owned by the same person or team. Otherwise, when the owner of {\ttfamily Foo} changes it, your test could break. (You can\textquotesingle{}t really expect {\ttfamily Foo}\textquotesingle{}s maintainer to fix every test that uses {\ttfamily Foo}, can you?)

So, the rule of thumb is\+: if you need to mock {\ttfamily Foo} and it\textquotesingle{}s owned by others, define the mock class in {\ttfamily Foo}\textquotesingle{}s package (better, in a {\ttfamily testing} sub-\/package such that you can clearly separate production code and testing utilities), and put it in a {\ttfamily mock\+\_\+foo.\+h}. Then everyone can reference {\ttfamily mock\+\_\+foo.\+h} from their tests. If {\ttfamily Foo} ever changes, there is only one copy of {\ttfamily \mbox{\hyperlink{class_mock_foo}{Mock\+Foo}}} to change, and only tests that depend on the changed methods need to be fixed.

Another way to do it\+: you can introduce a thin layer {\ttfamily Foo\+Adaptor} on top of {\ttfamily Foo} and code to this new interface. Since you own {\ttfamily Foo\+Adaptor}, you can absorb changes in {\ttfamily Foo} much more easily. While this is more work initially, carefully choosing the adaptor interface can make your code easier to write and more readable (a net win in the long run), as you can choose {\ttfamily Foo\+Adaptor} to fit your specific domain much better than {\ttfamily Foo} does.

\section*{Using Mocks in Tests}

Once you have a mock class, using it is easy. The typical work flow is\+:


\begin{DoxyEnumerate}
\item Import the Google \mbox{\hyperlink{class_mock}{Mock}} names from the {\ttfamily testing} namespace such that you can use them unqualified (You only have to do it once per file. Remember that namespaces are a good idea and good for your health.).
\end{DoxyEnumerate}
\begin{DoxyEnumerate}
\item Create some mock objects.
\end{DoxyEnumerate}
\begin{DoxyEnumerate}
\item Specify your expectations on them (How many times will a method be called? With what arguments? What should it do? etc.).
\end{DoxyEnumerate}
\begin{DoxyEnumerate}
\item Exercise some code that uses the mocks; optionally, check the result using Google Test assertions. If a mock method is called more than expected or with wrong arguments, you\textquotesingle{}ll get an error immediately.
\end{DoxyEnumerate}
\begin{DoxyEnumerate}
\item When a mock is destructed, Google \mbox{\hyperlink{class_mock}{Mock}} will automatically check whether all expectations on it have been satisfied.
\end{DoxyEnumerate}

Here\textquotesingle{}s an example\+:


\begin{DoxyCode}{0}
\DoxyCodeLine{\#include "path/to/mock-turtle.h"}
\DoxyCodeLine{\#include <gmock/gmock.h>}
\DoxyCodeLine{\#include <gtest/gtest.h>}
\DoxyCodeLine{using ::testing::AtLeast;                     // \#1}
\DoxyCodeLine{}
\DoxyCodeLine{TEST(PainterTest, CanDrawSomething) \{}
\DoxyCodeLine{  MockTurtle turtle;                          // \#2}
\DoxyCodeLine{  EXPECT\_CALL(turtle, PenDown())              // \#3}
\DoxyCodeLine{      .Times(AtLeast(1));}
\DoxyCodeLine{}
\DoxyCodeLine{  Painter painter(\&turtle);                   // \#4}
\DoxyCodeLine{}
\DoxyCodeLine{  EXPECT\_TRUE(painter.DrawCircle(0, 0, 10));}
\DoxyCodeLine{\}                                             // \#5}
\DoxyCodeLine{}
\DoxyCodeLine{int main(int argc, char** argv) \{}
\DoxyCodeLine{  // The following line must be executed to initialize Google Mock}
\DoxyCodeLine{  // (and Google Test) before running the tests.}
\DoxyCodeLine{  ::testing::InitGoogleMock(\&argc, argv);}
\DoxyCodeLine{  return RUN\_ALL\_TESTS();}
\DoxyCodeLine{\}}
\end{DoxyCode}


As you might have guessed, this test checks that {\ttfamily Pen\+Down()} is called at least once. If the {\ttfamily painter} object didn\textquotesingle{}t call this method, your test will fail with a message like this\+:


\begin{DoxyCode}{0}
\DoxyCodeLine{path/to/my\_test.cc:119: Failure}
\DoxyCodeLine{Actual function call count doesn't match this expectation:}
\DoxyCodeLine{Actually: never called;}
\DoxyCodeLine{Expected: called at least once.}
\end{DoxyCode}


{\bfseries{Tip 1\+:}} If you run the test from an Emacs buffer, you can hit {\ttfamily $<$Enter$>$} on the line number displayed in the error message to jump right to the failed expectation.

{\bfseries{Tip 2\+:}} If your mock objects are never deleted, the final verification won\textquotesingle{}t happen. Therefore it\textquotesingle{}s a good idea to use a heap leak checker in your tests when you allocate mocks on the heap.

{\bfseries{Important note\+:}} Google \mbox{\hyperlink{class_mock}{Mock}} requires expectations to be set {\bfseries{before}} the mock functions are called, otherwise the behavior is {\bfseries{undefined}}. In particular, you mustn\textquotesingle{}t interleave {\ttfamily E\+X\+P\+E\+C\+T\+\_\+\+C\+A\+L\+L()}s and calls to the mock functions.

This means {\ttfamily E\+X\+P\+E\+C\+T\+\_\+\+C\+A\+L\+L()} should be read as expecting that a call will occur {\itshape in the future}, not that a call has occurred. Why does Google \mbox{\hyperlink{class_mock}{Mock}} work like that? Well, specifying the expectation beforehand allows Google \mbox{\hyperlink{class_mock}{Mock}} to report a violation as soon as it arises, when the context (stack trace, etc) is still available. This makes debugging much easier.

Admittedly, this test is contrived and doesn\textquotesingle{}t do much. You can easily achieve the same effect without using Google \mbox{\hyperlink{class_mock}{Mock}}. However, as we shall reveal soon, Google \mbox{\hyperlink{class_mock}{Mock}} allows you to do {\itshape much more} with the mocks.

\subsection*{Using Google \mbox{\hyperlink{class_mock}{Mock}} with Any Testing Framework}

If you want to use something other than Google Test (e.\+g. \href{http://apps.sourceforge.net/mediawiki/cppunit/index.php?title=Main_Page}\texttt{ Cpp\+Unit} or \href{http://cxxtest.tigris.org/}\texttt{ Cxx\+Test}) as your testing framework, just change the {\ttfamily main()} function in the previous section to\+: 
\begin{DoxyCode}{0}
\DoxyCodeLine{int main(int argc, char** argv) \{}
\DoxyCodeLine{  // The following line causes Google Mock to throw an exception on failure,}
\DoxyCodeLine{  // which will be interpreted by your testing framework as a test failure.}
\DoxyCodeLine{  ::testing::GTEST\_FLAG(throw\_on\_failure) = true;}
\DoxyCodeLine{  ::testing::InitGoogleMock(\&argc, argv);}
\DoxyCodeLine{  ... whatever your testing framework requires ...}
\DoxyCodeLine{\}}
\end{DoxyCode}


This approach has a catch\+: it makes Google \mbox{\hyperlink{class_mock}{Mock}} throw an exception from a mock object\textquotesingle{}s destructor sometimes. With some compilers, this sometimes causes the test program to crash. You\textquotesingle{}ll still be able to notice that the test has failed, but it\textquotesingle{}s not a graceful failure.

A better solution is to use Google Test\textquotesingle{}s \href{http://code.google.com/p/googletest/wiki/GoogleTestAdvancedGuide#Extending_Google_Test_by_Handling_Test_Events}\texttt{ event listener A\+PI} to report a test failure to your testing framework properly. You\textquotesingle{}ll need to implement the {\ttfamily On\+Test\+Part\+Result()} method of the event listener interface, but it should be straightforward.

If this turns out to be too much work, we suggest that you stick with Google Test, which works with Google \mbox{\hyperlink{class_mock}{Mock}} seamlessly (in fact, it is technically part of Google \mbox{\hyperlink{class_mock}{Mock}}.). If there is a reason that you cannot use Google Test, please let us know.

\section*{Setting Expectations}

The key to using a mock object successfully is to set the {\itshape right expectations} on it. If you set the expectations too strict, your test will fail as the result of unrelated changes. If you set them too loose, bugs can slip through. You want to do it just right such that your test can catch exactly the kind of bugs you intend it to catch. Google \mbox{\hyperlink{class_mock}{Mock}} provides the necessary means for you to do it \char`\"{}just right.\char`\"{}

\subsection*{General Syntax}

In Google \mbox{\hyperlink{class_mock}{Mock}} we use the {\ttfamily E\+X\+P\+E\+C\+T\+\_\+\+C\+A\+L\+L()} macro to set an expectation on a mock method. The general syntax is\+:


\begin{DoxyCode}{0}
\DoxyCodeLine{EXPECT\_CALL(mock\_object, method(matchers))}
\DoxyCodeLine{    .Times(cardinality)}
\DoxyCodeLine{    .WillOnce(action)}
\DoxyCodeLine{    .WillRepeatedly(action);}
\end{DoxyCode}


The macro has two arguments\+: first the mock object, and then the method and its arguments. Note that the two are separated by a comma ({\ttfamily ,}), not a period ({\ttfamily .}). (Why using a comma? The answer is that it was necessary for technical reasons.)

The macro can be followed by some optional {\itshape clauses} that provide more information about the expectation. We\textquotesingle{}ll discuss how each clause works in the coming sections.

This syntax is designed to make an expectation read like English. For example, you can probably guess that


\begin{DoxyCode}{0}
\DoxyCodeLine{using ::testing::Return;...}
\DoxyCodeLine{EXPECT\_CALL(turtle, GetX())}
\DoxyCodeLine{    .Times(5)}
\DoxyCodeLine{    .WillOnce(Return(100))}
\DoxyCodeLine{    .WillOnce(Return(150))}
\DoxyCodeLine{    .WillRepeatedly(Return(200));}
\end{DoxyCode}


says that the {\ttfamily turtle} object\textquotesingle{}s {\ttfamily Get\+X()} method will be called five times, it will return 100 the first time, 150 the second time, and then 200 every time. Some people like to call this style of syntax a Domain-\/\+Specific Language (D\+SL).

{\bfseries{Note\+:}} Why do we use a macro to do this? It serves two purposes\+: first it makes expectations easily identifiable (either by {\ttfamily grep} or by a human reader), and second it allows Google \mbox{\hyperlink{class_mock}{Mock}} to include the source file location of a failed expectation in messages, making debugging easier.

\subsection*{Matchers\+: What Arguments Do We Expect?}

When a mock function takes arguments, we must specify what arguments we are expecting; for example\+:


\begin{DoxyCode}{0}
\DoxyCodeLine{// Expects the turtle to move forward by 100 units.}
\DoxyCodeLine{EXPECT\_CALL(turtle, Forward(100));}
\end{DoxyCode}


Sometimes you may not want to be too specific (Remember that talk about tests being too rigid? Over specification leads to brittle tests and obscures the intent of tests. Therefore we encourage you to specify only what\textquotesingle{}s necessary -\/ no more, no less.). If you care to check that {\ttfamily Forward()} will be called but aren\textquotesingle{}t interested in its actual argument, write {\ttfamily \+\_\+} as the argument, which means \char`\"{}anything goes\char`\"{}\+:


\begin{DoxyCode}{0}
\DoxyCodeLine{using ::testing::\_;}
\DoxyCodeLine{...}
\DoxyCodeLine{// Expects the turtle to move forward.}
\DoxyCodeLine{EXPECT\_CALL(turtle, Forward(\_));}
\end{DoxyCode}


{\ttfamily \+\_\+} is an instance of what we call {\bfseries{matchers}}. A matcher is like a predicate and can test whether an argument is what we\textquotesingle{}d expect. You can use a matcher inside {\ttfamily E\+X\+P\+E\+C\+T\+\_\+\+C\+A\+L\+L()} wherever a function argument is expected.

A list of built-\/in matchers can be found in the Cheat\+Sheet. For example, here\textquotesingle{}s the {\ttfamily Ge} (greater than or equal) matcher\+:


\begin{DoxyCode}{0}
\DoxyCodeLine{using ::testing::Ge;...}
\DoxyCodeLine{EXPECT\_CALL(turtle, Forward(Ge(100)));}
\end{DoxyCode}


This checks that the turtle will be told to go forward by at least 100 units.

\subsection*{Cardinalities\+: How Many Times Will It Be Called?}

The first clause we can specify following an {\ttfamily E\+X\+P\+E\+C\+T\+\_\+\+C\+A\+L\+L()} is {\ttfamily Times()}. We call its argument a {\bfseries{cardinality}} as it tells {\itshape how many times} the call should occur. It allows us to repeat an expectation many times without actually writing it as many times. More importantly, a cardinality can be \char`\"{}fuzzy\char`\"{}, just like a matcher can be. This allows a user to express the intent of a test exactly.

An interesting special case is when we say {\ttfamily Times(0)}. You may have guessed -\/ it means that the function shouldn\textquotesingle{}t be called with the given arguments at all, and Google \mbox{\hyperlink{class_mock}{Mock}} will report a Google Test failure whenever the function is (wrongfully) called.

We\textquotesingle{}ve seen {\ttfamily At\+Least(n)} as an example of fuzzy cardinalities earlier. For the list of built-\/in cardinalities you can use, see the Cheat\+Sheet.

The {\ttfamily Times()} clause can be omitted. {\bfseries{If you omit {\ttfamily Times()}, Google \mbox{\hyperlink{class_mock}{Mock}} will infer the cardinality for you.}} The rules are easy to remember\+:


\begin{DoxyItemize}
\item If {\bfseries{neither}} {\ttfamily Will\+Once()} {\bfseries{nor}} {\ttfamily Will\+Repeatedly()} is in the {\ttfamily E\+X\+P\+E\+C\+T\+\_\+\+C\+A\+L\+L()}, the inferred cardinality is {\ttfamily Times(1)}.
\item If there are {\ttfamily n Will\+Once()}\textquotesingle{}s but {\bfseries{no}} {\ttfamily Will\+Repeatedly()}, where {\ttfamily n} $>$= 1, the cardinality is {\ttfamily Times(n)}.
\item If there are {\ttfamily n Will\+Once()}\textquotesingle{}s and {\bfseries{one}} {\ttfamily Will\+Repeatedly()}, where {\ttfamily n} $>$= 0, the cardinality is {\ttfamily Times(\+At\+Least(n))}.
\end{DoxyItemize}

{\bfseries{Quick quiz\+:}} what do you think will happen if a function is expected to be called twice but actually called four times?

\subsection*{Actions\+: What Should It Do?}

Remember that a mock object doesn\textquotesingle{}t really have a working implementation? We as users have to tell it what to do when a method is invoked. This is easy in Google \mbox{\hyperlink{class_mock}{Mock}}.

First, if the return type of a mock function is a built-\/in type or a pointer, the function has a {\bfseries{default action}} (a {\ttfamily void} function will just return, a {\ttfamily bool} function will return {\ttfamily false}, and other functions will return 0). If you don\textquotesingle{}t say anything, this behavior will be used.

Second, if a mock function doesn\textquotesingle{}t have a default action, or the default action doesn\textquotesingle{}t suit you, you can specify the action to be taken each time the expectation matches using a series of {\ttfamily Will\+Once()} clauses followed by an optional {\ttfamily Will\+Repeatedly()}. For example,


\begin{DoxyCode}{0}
\DoxyCodeLine{using ::testing::Return;...}
\DoxyCodeLine{EXPECT\_CALL(turtle, GetX())}
\DoxyCodeLine{    .WillOnce(Return(100))}
\DoxyCodeLine{    .WillOnce(Return(200))}
\DoxyCodeLine{    .WillOnce(Return(300));}
\end{DoxyCode}


This says that {\ttfamily turtle.\+Get\+X()} will be called {\itshape exactly three times} (Google \mbox{\hyperlink{class_mock}{Mock}} inferred this from how many {\ttfamily Will\+Once()} clauses we\textquotesingle{}ve written, since we didn\textquotesingle{}t explicitly write {\ttfamily Times()}), and will return 100, 200, and 300 respectively.


\begin{DoxyCode}{0}
\DoxyCodeLine{using ::testing::Return;...}
\DoxyCodeLine{EXPECT\_CALL(turtle, GetY())}
\DoxyCodeLine{    .WillOnce(Return(100))}
\DoxyCodeLine{    .WillOnce(Return(200))}
\DoxyCodeLine{    .WillRepeatedly(Return(300));}
\end{DoxyCode}


says that {\ttfamily turtle.\+Get\+Y()} will be called {\itshape at least twice} (Google \mbox{\hyperlink{class_mock}{Mock}} knows this as we\textquotesingle{}ve written two {\ttfamily Will\+Once()} clauses and a {\ttfamily Will\+Repeatedly()} while having no explicit {\ttfamily Times()}), will return 100 the first time, 200 the second time, and 300 from the third time on.

Of course, if you explicitly write a {\ttfamily Times()}, Google \mbox{\hyperlink{class_mock}{Mock}} will not try to infer the cardinality itself. What if the number you specified is larger than there are {\ttfamily Will\+Once()} clauses? Well, after all {\ttfamily Will\+Once()}s are used up, Google \mbox{\hyperlink{class_mock}{Mock}} will do the {\itshape default} action for the function every time (unless, of course, you have a {\ttfamily Will\+Repeatedly()}.).

What can we do inside {\ttfamily Will\+Once()} besides {\ttfamily Return()}? You can return a reference using {\ttfamily Return\+Ref(variable)}, or invoke a pre-\/defined function, among others.

{\bfseries{Important note\+:}} The {\ttfamily E\+X\+P\+E\+C\+T\+\_\+\+C\+A\+L\+L()} statement evaluates the action clause only once, even though the action may be performed many times. Therefore you must be careful about side effects. The following may not do what you want\+:


\begin{DoxyCode}{0}
\DoxyCodeLine{int n = 100;}
\DoxyCodeLine{EXPECT\_CALL(turtle, GetX())}
\DoxyCodeLine{.Times(4)}
\DoxyCodeLine{.WillOnce(Return(n++));}
\end{DoxyCode}


Instead of returning 100, 101, 102, ..., consecutively, this mock function will always return 100 as {\ttfamily n++} is only evaluated once. Similarly, {\ttfamily Return(new Foo)} will create a new {\ttfamily Foo} object when the {\ttfamily E\+X\+P\+E\+C\+T\+\_\+\+C\+A\+L\+L()} is executed, and will return the same pointer every time. If you want the side effect to happen every time, you need to define a custom action, which we\textquotesingle{}ll teach in the Cook\+Book.

Time for another quiz! What do you think the following means?


\begin{DoxyCode}{0}
\DoxyCodeLine{using ::testing::Return;...}
\DoxyCodeLine{EXPECT\_CALL(turtle, GetY())}
\DoxyCodeLine{.Times(4)}
\DoxyCodeLine{.WillOnce(Return(100));}
\end{DoxyCode}


Obviously {\ttfamily turtle.\+Get\+Y()} is expected to be called four times. But if you think it will return 100 every time, think twice! Remember that one {\ttfamily Will\+Once()} clause will be consumed each time the function is invoked and the default action will be taken afterwards. So the right answer is that {\ttfamily turtle.\+Get\+Y()} will return 100 the first time, but {\bfseries{return 0 from the second time on}}, as returning 0 is the default action for {\ttfamily int} functions.

\subsection*{Using Multiple Expectations}

So far we\textquotesingle{}ve only shown examples where you have a single expectation. More realistically, you\textquotesingle{}re going to specify expectations on multiple mock methods, which may be from multiple mock objects.

By default, when a mock method is invoked, Google \mbox{\hyperlink{class_mock}{Mock}} will search the expectations in the {\bfseries{reverse order}} they are defined, and stop when an active expectation that matches the arguments is found (you can think of it as \char`\"{}newer rules override older ones.\char`\"{}). If the matching expectation cannot take any more calls, you will get an upper-\/bound-\/violated failure. Here\textquotesingle{}s an example\+:


\begin{DoxyCode}{0}
\DoxyCodeLine{using ::testing::\_;...}
\DoxyCodeLine{EXPECT\_CALL(turtle, Forward(\_));  // \#1}
\DoxyCodeLine{EXPECT\_CALL(turtle, Forward(10))  // \#2}
\DoxyCodeLine{    .Times(2);}
\end{DoxyCode}


If {\ttfamily Forward(10)} is called three times in a row, the third time it will be an error, as the last matching expectation (\#2) has been saturated. If, however, the third {\ttfamily Forward(10)} call is replaced by {\ttfamily Forward(20)}, then it would be OK, as now \#1 will be the matching expectation.

{\bfseries{Side note\+:}} Why does Google \mbox{\hyperlink{class_mock}{Mock}} search for a match in the {\itshape reverse} order of the expectations? The reason is that this allows a user to set up the default expectations in a mock object\textquotesingle{}s constructor or the test fixture\textquotesingle{}s set-\/up phase and then customize the mock by writing more specific expectations in the test body. So, if you have two expectations on the same method, you want to put the one with more specific matchers {\bfseries{after}} the other, or the more specific rule would be shadowed by the more general one that comes after it.

\subsection*{Ordered vs Unordered Calls}

By default, an expectation can match a call even though an earlier expectation hasn\textquotesingle{}t been satisfied. In other words, the calls don\textquotesingle{}t have to occur in the order the expectations are specified.

Sometimes, you may want all the expected calls to occur in a strict order. To say this in Google \mbox{\hyperlink{class_mock}{Mock}} is easy\+:


\begin{DoxyCode}{0}
\DoxyCodeLine{using ::testing::InSequence;...}
\DoxyCodeLine{TEST(FooTest, DrawsLineSegment) \{}
\DoxyCodeLine{  ...}
\DoxyCodeLine{  \{}
\DoxyCodeLine{    InSequence dummy;}
\DoxyCodeLine{}
\DoxyCodeLine{    EXPECT\_CALL(turtle, PenDown());}
\DoxyCodeLine{    EXPECT\_CALL(turtle, Forward(100));}
\DoxyCodeLine{    EXPECT\_CALL(turtle, PenUp());}
\DoxyCodeLine{  \}}
\DoxyCodeLine{  Foo();}
\DoxyCodeLine{\}}
\end{DoxyCode}


By creating an object of type {\ttfamily In\+Sequence}, all expectations in its scope are put into a {\itshape sequence} and have to occur {\itshape sequentially}. Since we are just relying on the constructor and destructor of this object to do the actual work, its name is really irrelevant.

In this example, we test that {\ttfamily Foo()} calls the three expected functions in the order as written. If a call is made out-\/of-\/order, it will be an error.

(What if you care about the relative order of some of the calls, but not all of them? Can you specify an arbitrary partial order? The answer is ... yes! If you are impatient, the details can be found in the Cook\+Book.)

\subsection*{All Expectations Are Sticky (Unless Said Otherwise)}

Now let\textquotesingle{}s do a quick quiz to see how well you can use this mock stuff already. How would you test that the turtle is asked to go to the origin {\itshape exactly twice} (you want to ignore any other instructions it receives)?

After you\textquotesingle{}ve come up with your answer, take a look at ours and compare notes (solve it yourself first -\/ don\textquotesingle{}t cheat!)\+:


\begin{DoxyCode}{0}
\DoxyCodeLine{using ::testing::\_;...}
\DoxyCodeLine{EXPECT\_CALL(turtle, GoTo(\_, \_))  // \#1}
\DoxyCodeLine{    .Times(AnyNumber());}
\DoxyCodeLine{EXPECT\_CALL(turtle, GoTo(0, 0))  // \#2}
\DoxyCodeLine{    .Times(2);}
\end{DoxyCode}


Suppose {\ttfamily turtle.\+Go\+To(0, 0)} is called three times. In the third time, Google \mbox{\hyperlink{class_mock}{Mock}} will see that the arguments match expectation \#2 (remember that we always pick the last matching expectation). Now, since we said that there should be only two such calls, Google \mbox{\hyperlink{class_mock}{Mock}} will report an error immediately. This is basically what we\textquotesingle{}ve told you in the \char`\"{}\+Using Multiple Expectations\char`\"{} section above.

This example shows that {\bfseries{expectations in Google \mbox{\hyperlink{class_mock}{Mock}} are \char`\"{}sticky\char`\"{} by default}}, in the sense that they remain active even after we have reached their invocation upper bounds. This is an important rule to remember, as it affects the meaning of the spec, and is {\bfseries{different}} to how it\textquotesingle{}s done in many other mocking frameworks (Why\textquotesingle{}d we do that? Because we think our rule makes the common cases easier to express and understand.).

Simple? Let\textquotesingle{}s see if you\textquotesingle{}ve really understood it\+: what does the following code say?


\begin{DoxyCode}{0}
\DoxyCodeLine{using ::testing::Return;}
\DoxyCodeLine{...}
\DoxyCodeLine{for (int i = n; i > 0; i--) \{}
\DoxyCodeLine{  EXPECT\_CALL(turtle, GetX())}
\DoxyCodeLine{      .WillOnce(Return(10*i));}
\DoxyCodeLine{\}}
\end{DoxyCode}


If you think it says that {\ttfamily turtle.\+Get\+X()} will be called {\ttfamily n} times and will return 10, 20, 30, ..., consecutively, think twice! The problem is that, as we said, expectations are sticky. So, the second time {\ttfamily turtle.\+Get\+X()} is called, the last (latest) {\ttfamily E\+X\+P\+E\+C\+T\+\_\+\+C\+A\+L\+L()} statement will match, and will immediately lead to an \char`\"{}upper bound exceeded\char`\"{} error -\/ this piece of code is not very useful!

One correct way of saying that {\ttfamily turtle.\+Get\+X()} will return 10, 20, 30, ..., is to explicitly say that the expectations are {\itshape not} sticky. In other words, they should {\itshape retire} as soon as they are saturated\+:


\begin{DoxyCode}{0}
\DoxyCodeLine{using ::testing::Return;}
\DoxyCodeLine{...}
\DoxyCodeLine{for (int i = n; i > 0; i--) \{}
\DoxyCodeLine{  EXPECT\_CALL(turtle, GetX())}
\DoxyCodeLine{    .WillOnce(Return(10*i))}
\DoxyCodeLine{    .RetiresOnSaturation();}
\DoxyCodeLine{\}}
\end{DoxyCode}


And, there\textquotesingle{}s a better way to do it\+: in this case, we expect the calls to occur in a specific order, and we line up the actions to match the order. Since the order is important here, we should make it explicit using a sequence\+:


\begin{DoxyCode}{0}
\DoxyCodeLine{using ::testing::InSequence;}
\DoxyCodeLine{using ::testing::Return;}
\DoxyCodeLine{...}
\DoxyCodeLine{\{}
\DoxyCodeLine{  InSequence s;}
\DoxyCodeLine{}
\DoxyCodeLine{  for (int i = 1; i <= n; i++) \{}
\DoxyCodeLine{    EXPECT\_CALL(turtle, GetX())}
\DoxyCodeLine{        .WillOnce(Return(10*i))}
\DoxyCodeLine{        .RetiresOnSaturation();}
\DoxyCodeLine{  \}}
\DoxyCodeLine{\}}
\end{DoxyCode}


By the way, the other situation where an expectation may {\itshape not} be sticky is when it\textquotesingle{}s in a sequence -\/ as soon as another expectation that comes after it in the sequence has been used, it automatically retires (and will never be used to match any call).

\subsection*{Uninteresting Calls}

A mock object may have many methods, and not all of them are that interesting. For example, in some tests we may not care about how many times {\ttfamily Get\+X()} and {\ttfamily Get\+Y()} get called.

In Google \mbox{\hyperlink{class_mock}{Mock}}, if you are not interested in a method, just don\textquotesingle{}t say anything about it. If a call to this method occurs, you\textquotesingle{}ll see a warning in the test output, but it won\textquotesingle{}t be a failure.

\section*{What Now?}

Congratulations! You\textquotesingle{}ve learned enough about Google \mbox{\hyperlink{class_mock}{Mock}} to start using it. Now, you might want to join the \href{http://groups.google.com/group/googlemock}\texttt{ googlemock} discussion group and actually write some tests using Google \mbox{\hyperlink{class_mock}{Mock}} -\/ it will be fun. Hey, it may even be addictive -\/ you\textquotesingle{}ve been warned.

Then, if you feel like increasing your mock quotient, you should move on to the Cook\+Book. You can learn many advanced features of Google \mbox{\hyperlink{class_mock}{Mock}} there -- and advance your level of enjoyment and testing bliss. 